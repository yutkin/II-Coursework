\section*{Аннотация}
Работа посвящена решению задачи распознавания образов с использованием глубоких свёрточных 
нейронных сетей. В результате исследования было обучено несколько нейронных сетей различной глубины 
(от 9 до 18 слоёв), часть из которых обучалась на предварительно обработанных данных. Максимальная 
точность одиночной модели составила 93,4\%, объединение нейронных сетей в ансамбль позволило
увеличилась точность распознавания до 94,15\%, что сравнимо с точностью человека. Все эксперименты 
проводились на наборе изображений CIFAR-10, с использованием графических процессоров и фреймворка
для глубокого обучения Caffe.

\section*{Abstract}
This research dedicated to application of deep convolutional neural networks for
solving object recognition task. As a result, several models of different depth (from 9 to 18 layers)
were trained on preprocessed and augmented data. The best model has shown accuracy of 93,4\% on a test images.
Whereas, ensemble of combined models has shown accuracy of 94,15\% which is comparable with human performance
on this data. All experiments were carried out on CIFAR-10 dataset with modern GPUs and Caffe framework for deep learning.