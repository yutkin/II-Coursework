\section*{Аннотация}
Работа посвящена решению задачи распознавания образов с использованием глубоких свёрточных 
нейронных сетей. В результате исследования было обучено несколько нейронных сетей различной глубины 
(от 9 до 18 слоёв), часть из которых обучалась на предварительно обработанных данных. Максимальная 
точность одиночной модели составила 93,4\%, объединение нейронных сетей в ансамбль позволило
увеличилась точность распознавания до 94,15\%, что сравнимо с точностью человека. Все эксперименты 
проводились на наборе изображений CIFAR-10, с использованием графических процессоров и фреймворка
для глубокого обучения Caffe.

\section*{Abstract}
The research focused on application of deep convolutional neural networks for
solving object recognition task in natural images. As a result, several models of different depth
(from 9 to 18 layers) have been trained on preprocessed and augmented data. The best model has shown accuracy 
of 93,4\% on a test images. Whereas, ensemble of combined models has achived accuracy of 94,15\% which is 
close-to-human performance. All experiments have been carried out on the CIFAR-10 dataset with modern GPUs
and Caffe framework for a deep learning.