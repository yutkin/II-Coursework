\section{Введение}
В последние годы компьютерное зрение является одной из самых активно развивающихся областей 
искусственного интеллекта, основная задача которой~--- научить компьютеры воспроизводить 
зрительные способности человека. Примером такой задачи является распознавание образов.

До недавнего времени распознавание объектов на изображении проводилось с помощью алгоритмов, 
для которых вручную приходилось проектировать признаки (feature engineering). 
Основной недостаток такого подхода --- невозможность находить в изображении 
средние и многоуровневые абстракции, такие как части объекта или пересечения различных краёв. 
Кроме того, признаки созданные вручную для одного типа изображений зачастую не были применимы к 
другим. Однако, недавние разработки в области машинного обучения, известные как <<глубокое 
обучение>> (deep learning), показали, как вычислительные модели, состоящие из множества слоёв, 
могут автоматически изучать иерархии признаков прямо из набора данных.

На сегодняшний день глубокое обучение развилось и укрепилось в отдельную ветвь машинного обучения, 
алгоритмы которой способствовали значительному улучшению результатов в различных задачах, таких как 
обработка естественного языка, распознавание визуальных образов и др. Благодаря технологиям 
глубокого обучения сегодня мы можем искать похожие фотографии в Google Photos, или восхищаться 
беспилотными автомобилями, которые управляются искусственным интеллектом.

Свёрточные нейронные сети (СНН) --- одна из главных причин прорыва в компьютерном зрении. 
Изначально представленные в 1980 году Кунихикой Фукусимой как NeoCognitron \cite{Neocognitron}, а 
затем улучшенные в 1998 году Яном Лекуном до LeNet-5 \cite{lecun-98}, СНН получили славу благодаря 
впечатляющему успеху в  распознавании рукописных цифр. Для следующего прорыва в компьютерном зрении 
потребовалось чуть больше двадцати лет. С увеличением производительности графических процессоров 
(GPU), стало возможным обучение глубоких нейронных сетей. В 2012 году глубокая СНН победила в 
мировом соревновании ImageNet Large-scale Visual Recognition Challenge (ILSVRC), значительно 
превзойдя предыдущие результаты, достигнутые алгоритмами полагающимися на ручную генерацию 
признаков.