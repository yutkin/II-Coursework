\section{Введение}
Распознавание образов является развивающийся и перспективной областью компьютерного зрения. На 
сегодняшний день технологии распознавания встречаются в различных сферах деятельности человека, 
начиная от медицины, где системы компьютерной диагностики способны детектировать раковые опухоли, и 
заканчивая беспилотными автомобилями, которые управляются искусственным интеллектом.

До недавнего времени распознавание объектов на изображении проводилось с помощью SIFT, HOG и др., 
однако, для таких алгоритмов приходилось вручную проектировать признаки (feature engineering). 
Такой подход имел недостатки, главный из которых --- невозможность находить в изображении 
абстракции среднего или высокого уровня, такие как части объекта или пересечения различных краёв. 
Кроме того, признаки созданные вручную для одного типа изображений зачастую не были применимы к 
другим типам. Однако, недавние разработки в области машинного обучения, известные как <<глубокое 
обучение>> (deep learning), показали, как вычислительные модели, состоящие из множества слоёв, 
могут автоматически изучать иерархии признаков прямо из набора данных.
На сегодняшний день глубокое обучение развилось и укрепилось в отдельную ветвь машинного обучения, 
алгоритмы которой способствовали значительному улучшению результатов в различных задачах, таких как 
обработка естественного языка, распознавание визуальных образов и др. Благодаря технологиям 
глубокого обучения сегодня мы можем искать похожие фотографии в Google Photos, или восхищаться тем, 
как Facebook описывает содержимое фотографий для людей с ограниченными возможностями зрения.

Свёрточные нейронные сети (СНН) --- одна из главных причин прорыва в компьютерном зрении. 
Изначально представленные в 1980 году Кунихикой Фукусимой как NeoCognitron, а затем улучшенные в 
1998 году Яном Лекуном до LeNet-5, СНН получили славу благодаря впечатляющему успеху в 
распознавании рукописных цифр. Для следующего прорыва в компьютерном зрении потребовалось чуть 
больше двадцати лет. С увеличением производительности графических процессоров (GPU), стало 
возможным обучение глубоких нейронных сетей. В 2012 году глубокая СНН победила в мировом 
соревновании ImageNet Large-scale Visual Recognition Challenge (ILSVRC), значительно превзойдя 
предыдущие результаты, достигнутые алгоритмами полагающимися на ручную генерацию признаков.