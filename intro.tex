\section{Введение}
В последние годы компьютерное зрение является одной из самых активно развивающихся областей
искусственного интеллекта. Подобный интерес к области обусловлен недавними успехами в обучении
компьютера воспроизводить зрительные способности человека, например, распознавать и классифицировать образы.

До недавнего времени распознавание объектов осуществлялось с помощью 
алгоритмов, для которых вручную приходилось проектировать признаки (feature engineering). 
Основной недостаток такого подхода "--- невозможность находить в изображении 
средние и многоуровневые абстракции, такие как части объекта или пересечения различных краёв. 
Кроме того, признаки созданные вручную для одного типа изображений зачастую не были применимы к 
другим. Однако, недавние разработки в области машинного обучения, известные как глубокое 
обучение (deep learning)\cite{brief_deep_learning}, показали, как вычислительные модели, состоящие из множества слоёв, 
могут автоматически изучать иерархии признаков из набора данных.

На сегодняшний день глубокое обучение развилось и укрепилось в отдельную ветвь машинного обучения, 
алгоритмы которой способствовали значительному улучшению результатов в различных задачах, таких как 
обработка естественного языка, распознавание визуальных образов и др.

Свёрточные нейронные сети (СНН) "--- одна из главных причин прорыва в компьютерном зрении. 
Изначально представленные в 1980 году Кунихикой Фукусимой как NeoCognitron \cite{Neocognitron}, а 
затем улучшенные в 1998 году Яном Лекуном до LeNet-5 \cite{lecun-98}, СНН получили славу благодаря 
впечатляющему успеху в  распознавании рукописных цифр. Для следующего прорыва в компьютерном зрении 
потребовалось чуть больше двадцати лет. С увеличением производительности графических процессоров 
(GPU), стало возможным обучение глубоких нейронных сетей. В 2012 году глубокая СНН победила в 
мировом соревновании ImageNet Large-scale Visual Recognition Challenge (ILSVRC), значительно 
превзойдя предыдущие результаты, достигнутые алгоритмами полагающимися на ручную генерацию 
признаков \cite{NIPS2012_4824}.

Таким образом, темой данного исследования является разработка системы распознавания объектов
(object recognition), основанной на глубоких свёрточных нейронных сетях.
Основные цели исследования заключались в:
\begin{enumerate}
    \item Изучении и применении на практике свёрточных нейронных сетей в решении задачи распознавания объектов;
    \item Изучении современных технологий машинного обучения, анализа данных и глубокого обучения.
\end{enumerate}
Для достижения поставленных целей были сформулированы следующие задачи:
\begin{enumerate}
    \item Проведение обзора статей и литературы, посвящённых свёрточным нейронным сетям и решению
    задачи распознавания образов;
    \item Выбор и изучение программного обеспечения для глубокого обучения;
    \item Выбор набора данных, на котором будут проводиться эксперименты;
    \item Проектирование и обучение нейронных сетей различных архитектур;
    \item Оценка и сравнение точностей полученных нейронных сетей;
    \item Объединение отдельно обученных моделей в ансамбль.
\end{enumerate}
Все поставленные задачи выполнены, цели исследования достигнуты. Результаты каждой из задач будут 
описаны далее, в основной части работы.
