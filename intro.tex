\section{Введение}
В последние годы компьютерное зрение является одной из самых активно развивающихся областей
искусственного интеллекта. Подобный интерес к области обусловлен недавними успехами в решении
такой задачи, как обучение компьютера воспроизводить зрительные способности человека, например,
распознавать и классифицировать образы.

До недавнего времени распознавание объектов осуществлялось с помощью 
алгоритмов, для которых вручную приходилось проектировать признаки (feature engineering). 
Основной недостаток такого подхода "--- невозможность находить в изображении 
средние и многоуровневые абстракции, такие как части объекта или пересечения различных краёв. 
Кроме того, признаки созданные вручную для одного типа изображений зачастую не были применимы к 
другим. Однако, недавние разработки в области машинного обучения, известные как <<глубокое 
обучение>> (deep learning), показали, как вычислительные модели, состоящие из множества слоёв, 
могут автоматически изучать иерархии признаков прямо из набора данных.

На сегодняшний день глубокое обучение развилось и укрепилось в отдельную ветвь машинного обучения, 
алгоритмы которой способствовали значительному улучшению результатов в различных задачах, таких как 
обработка естественного языка, распознавание визуальных образов и др. Благодаря технологиям 
глубокого обучения сегодня мы можем искать похожие фотографии в Google Photos, или восхищаться 
беспилотными автомобилями, которые управляются искусственным интеллектом.

Свёрточные нейронные сети (СНН) "--- одна из главных причин прорыва в компьютерном зрении. 
Изначально представленные в 1980 году Кунихикой Фукусимой как NeoCognitron \cite{Neocognitron}, а 
затем улучшенные в 1998 году Яном Лекуном до LeNet-5 \cite{lecun-98}, СНН получили славу благодаря 
впечатляющему успеху в  распознавании рукописных цифр. Для следующего прорыва в компьютерном зрении 
потребовалось чуть больше двадцати лет. С увеличением производительности графических процессоров 
(GPU), стало возможным обучение глубоких нейронных сетей. В 2012 году глубокая СНН победила в 
мировом соревновании ImageNet Large-scale Visual Recognition Challenge (ILSVRC), значительно 
превзойдя предыдущие результаты, достигнутые алгоритмами полагающимися на ручную генерацию 
признаков \cite{NIPS2012_4824}.

Таким образом, темой данного исследования является разработка системы распознавания объектов 
(object recognition), основанной на глубоких свёрточных нейронных сетях. Задача решалась с помощью  
фреймворка для глубокого обучения Caffe \cite{jia2014caffe} на наборе данных CIFAR-10 
\cite{learningmultiple}. Основные цели исследования заключались в следующем:
\begin{enumerate}
    \item Изучить и применить на практике свёрточные нейронные сети в решении задачи распознавания 
    объектов;
    \item Изучить современные технологии машинного обучения, анализа данных и глубокого  
    обучения.
\end{enumerate}
Для достижения поставленных целей были сформулированы следующие задачи:
\begin{enumerate}
    \item Провести обзор статей и литературы, посвящённых свёрточным нейронным сетям и решению 
    задачи распознавания образов;
    \item Выбрать и изучить программное обеспечение для глубокого обучения;
    \item Подобрать набор данных, на котором будут проводиться эксперименты;
    \item Спроектировать и обучить нейронные сети различных архитектур;
    \item Оценить и сравнить точности полученных нейронных сетей;
    \item Объединить отдельно обученные модели в ансамбль, с целью увеличения точности 
    распознавания.
\end{enumerate}
Все поставленные задачи выполнены, цели исследования достигнуты. Результаты каждой из задач будут 
описаны далее, в основной части работы.
