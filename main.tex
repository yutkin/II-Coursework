\documentclass[a4paper, 14pt]{extarticle}
\usepackage{geometry}

\usepackage{cmap} % Улучшенный поиск русских слов в полученном pdf-файле
\usepackage{mathtext} % русские буквы в формулах
\defaulthyphenchar=127 % Если стоит до fontenc, то переносы не впишутся в выделяемый текст при 
%копировании его в буфер обмена
\usepackage[T2A]{fontenc}
\usepackage[utf8]{inputenc}
\usepackage[english, russian]{babel}
\usepackage{pscyr}  
\renewcommand{\rmdefault}{ftm} % ftm - (TimesNewRoman), fac - Academy, fad - Advertisement, flz - 
%Lazurski, fcr - CourierNewPSM, others in pscyr.sty

\usepackage{amsthm,amsfonts,amsmath,amssymb,amscd} % Математические дополнения от AMS
\usepackage{mathtools} % Добавляет окружение multlined

\usepackage{longtable} % Длинные таблицы
\usepackage{multirow,makecell,array} % Улучшенное форматирование таблиц
\usepackage{booktabs} % Возможность оформления таблиц в классическом книжном стиле

\usepackage{soulutf8} % Поддержка переносоустойчивых подчёркиваний и зачёркиваний
\usepackage{icomma} % Запятая в десятичных дробях

\usepackage[usenames,dvipsnames,svgnames,table,rgb]{xcolor}

\usepackage{hyperref}

\usepackage{graphicx} % Подключаем пакет работы с графикой
\graphicspath{{../images/}{images/}} % Пути к изображениям

%%% Подписи %%%
\usepackage[singlelinecheck=off,center]{caption}
\usepackage{subcaption}

\usepackage[onehalfspacing]{setspace}

%%% Списки %%%
\usepackage{enumitem}

%%% Библиография %%%
\usepackage{cite} % Красивые ссылки на литературу

%%% Оглавление %%%
\usepackage[nottoc]{tocbibind}
\usepackage{tocloft}
\usepackage{titlesec}

\usepackage{abstract}
\usepackage{titlesec} % Растояние между заголовками и текстом

\geometry{a4paper,top=2cm,bottom=2cm,left=2cm,right=1cm}
%%% Выравнивание и переносы %%%
\sloppy                             % Избавляемся от переполнений
\clubpenalty=10000                  % Запрещаем разрыв страницы после первой строки абзаца
\widowpenalty=10000                 % Запрещаем разрыв страницы после последней строки абзаца
\usepackage{indentfirst}
\frenchspacing
\setlength{\parindent}{2.5em} % Абзацный отступ
\linespread{1.3}

% colors
\definecolor{linkcolor}{rgb}{0.9,0,0}
\definecolor{citecolor}{rgb}{0,0.6,0}
\definecolor{urlcolor}{rgb}{0,0,1}
\definecolor{mygreen}{rgb}{0,0.6,0}
\definecolor{mygray}{rgb}{0.5,0.5,0.5}
\definecolor{mymauve}{rgb}{0.58,0,0.82}

\hypersetup{				% Гиперссылки
    unicode=true,          % non-Latin characters in Acrobat’s bookmarks
    pdftoolbar=true,        % show Acrobat’s toolbar?
    pdfmenubar=true,        % show Acrobat’s menu?
    pdffitwindow=false,     % window fit to page when opened
    pdfstartview={FitH},    % fits the width of the page to the window
    pdftitle={Применение глубоких свёрточных нейронных сетей для решения задачи распознавания образов},    % title
    pdfauthor={Юткин Дмитрий Игоревич},     % author
    pdfsubject={Курсовая},   % subject of the document
    pdfcreator={Юткин Дмитрий Игоревич},   % creator of the document
    pdfproducer={Юткин Дмитрий Игоревич}, % producer of the document
    pdfkeywords={convolutional} {neural} {networks} 
    {deeplearning} {deep} {learning} {caffe} {сверточные} {нейронные} {сети}
    {глубокое} {обучение} {компьютерное} {зрение}, % Ключевые слова
    pdfnewwindow=true,
    pdflang={ru},
    linktocpage=true,
    plainpages=false,
    colorlinks,       	    % false: ссылки в рамках; true: цветные ссылки
    linkcolor={linkcolor},      % цвет ссылок типа ref, eqref и подобных
    citecolor={citecolor},      % цвет ссылок-цитат
    urlcolor={urlcolor},        % цвет гиперссылок
}

%% Рисунки %%%
\DeclareCaptionLabelSeparator{emdash}{~--- }
\captionsetup[figure]{labelsep=emdash,font=onehalfspacing,position=bottom}
\captionsetup{%
    singlelinecheck=off,                % Многострочные подписи, например у таблиц
    skip=2pt,                           % Вертикальная отбивка между подписью и содержимым рисунка или таблицы определяется ключом
    justification=centering,            % Центрирование подписей, заданных командой \caption
}

%% Подписи подрисунков %%%
\renewcommand{\thesubfigure}{\asbuk{subfigure}}           % Буквенные номера подрисунков
\captionsetup[subfigure]{font={normalsize},               % Шрифт подписи названий подрисунков (не отличается от основного)
    labelformat=brace,                                    % Формат обозначения подрисунка
    justification=centering,                              % Выключка подписей (форматирование), один из вариантов            
}

%%% Списки %%%
% Используем короткое тире (endash) для ненумерованных списков (ГОСТ 2.105-95, пункт 4.1.7, требует дефиса, но так лучше смотрится)
\renewcommand{\labelitemi}{\normalfont\bfseries{--}}

% Перечисление строчными буквами русского алфавита (ГОСТ 2.105-95, 4.1.7)
\makeatletter
    \AddEnumerateCounter{\asbuk}{\russian@alph}{щ}
\makeatother
\setlist[enumerate,1]{label=\asbuk{enumi})} % первого уровня 1), 2)...
\setlist[enumerate,2]{label=\arabic*)} % второго уровня а), б) ... 

\setlist{nosep,%                                    % Единый стиль для всех списков (пакет enumitem), без дополнительных интервалов.
    labelindent=\parindent,leftmargin=*%            % Каждый пункт, подпункт и перечисление записывают с абзацного отступа
}

%%% Переопределение именований %%%
\addto\captionsrussian{%
    \renewcommand{\partname}{Часть}
    \renewcommand{\abstractname}{Аннотация}
    \renewcommand{\contentsname}{Оглавление} % (ГОСТ Р 7.0.11-2011, 4)
    \renewcommand{\figurename}{Рисунок} % (ГОСТ Р 7.0.11-2011, 5.3.9)
    \renewcommand{\tablename}{Таблица} % (ГОСТ Р 7.0.11-2011, 5.3.10)
    \renewcommand{\indexname}{Предметный указатель}
    \renewcommand{\listfigurename}{Список рисунков}
    \renewcommand{\listtablename}{Список таблиц}
}
%
%%% Старт отсчет страниц с титульника
\makeatletter
\renewenvironment{titlepage}
 {%
  \if@twocolumn
    \@restonecoltrue\onecolumn
  \else
    \@restonecolfalse\newpage
  \fi
  \thispagestyle{empty}%
 }
 {%
  \if@restonecol
    \twocolumn
  \else
    \newpage
  \fi
 }
\makeatother
%%%

%%% Библиография %%%
\makeatletter
\bibliographystyle{utf8gost71u}     % Оформляем библиографию по ГОСТ 7.1 (ГОСТ Р 7.0.11-2011, 5.6.7)
\renewcommand{\@biblabel}[1]{#1.}   % Заменяем библиографию с квадратных скобок на точку
\makeatother

%%% Оглавление %%%
\cftsetrmarg{2.55em plus1fil} %To have the (sectional) titles in the ToC, etc., typeset ragged right with no hyphenation
\renewcommand{\cftsecdotsep}{\cftdotsep} % отбивка точками до номера страницы начала главы/раздела
\renewcommand{\cftsecpagefont}{\normalfont}        % нежирные номера страниц у глав в оглавлении
\renewcommand{\cftsecleader}{\cftdotfill{\cftsecdotsep}}% нежирные точки до номеров страниц у глав в оглавлении
\renewcommand{\cfttoctitlefont}{\filright\fontsize{16pt}{18pt}\selectfont\bfseries} % Размер заголовка оглавления

%\renewcommand{\cftsecfont}{}                       % нежирные названия глав в оглавлении
%\renewcommand\cftsecaftersnum{.\ }   % добавляет точку с пробелом после номера раздела в оглавлении
%\renewcommand\cftsecaftersnum{.\ }    % добавляет точку с пробелом после номера подраздела в оглавлении
%\renewcommand\cftsubsecaftersnum{.\ } % добавляет точку с пробелом после номера подподраздела в оглавлении
%\renewcommand\cftsecaftersnum{\quad}     % добавляет \quad после номера раздела в оглавлении
%\renewcommand\cftsecaftersnum{\quad}      % добавляет \quad после номера подраздела в оглавлении
%\renewcommand\cftsubsecaftersnum{\quad}   % добавляет \quad после номера подподраздела в оглавлении
%\addtocontents{toc}{~\hfill{Стр.}\par}% добавить Стр. над номерами страниц

%%% Оформление заголовков глав, разделов, подразделов %%%
\titleformat{\section}[block]
    %{\filcenter\fontsize{16pt}{18pt}\selectfont\bfseries}{\thesection\cftsecaftersnum}{0.5em}{} % по центру
    {\filright\fontsize{16pt}{18pt}\selectfont\bfseries}{\thesection\cftsecaftersnum}{0.5em}{} % справа

\titleformat{\subsection}[block]
    %{\filcenter\fontsize{16pt}{18pt}\selectfont\bfseries}{\thesubsection\cftsubsecaftersnum}{0.5em}{}
    {\filright\fontsize{16pt}{18pt}\selectfont\bfseries}{\thesubsection\cftsubsecaftersnum}{0.5em}{}

\titleformat{\subsubsection}[block]
    %{\filcenter\fontsize{16pt}{18pt}\selectfont\bfseries}{\thesubsubsection\cftsubsecaftersnum}{0.5em}{}
    {\filright\fontsize{16pt}{18pt}\selectfont\bfseries}{\thesubsubsection\cftsubsecaftersnum}{0.5em}{}

%\renewcommand{\abstractnamefont}{\fontsize{16pt}{18pt}\selectfont\bfseries} % Размер заголовка аннтоации


% Расстояние между текстом и заголовками
%\setlength{\abstitleskip}{-25pt} % расстояние между заголовком аннтоации и тестом
\titlespacing{\section}{\parindent}{*2}{*1} % Расстояние между заголовком раздела и текстом должно быть равно удвоенному межстрочному интервалу.  Расстояние между основаниями строк заголовка принимают такими же, как в тексте
\titlespacing{\subsection}{\parindent}{*2}{*1}
\titlespacing{\subsubsection}{\parindent}{*2}{*1}

% Listings
\lstset{ %
  backgroundcolor=\color{white},   % choose the background color; you must add \usepackage{color} or \usepackage{xcolor}
  basicstyle=\ttfamily\footnotesize,        % the size of the fonts that are used for the code
  breakatwhitespace=false,         % sets if automatic breaks should only happen at whitespace
  breaklines=true,                 % sets automatic line breaking
  captionpos=b,                    % sets the caption-position to bottom
  commentstyle=\ttfamily,    % comment style
  columns=fixed,
  deletekeywords={...},            % if you want to delete keywords from the given language
  escapeinside={\%*}{*)},          % if you want to add LaTeX within your code
  extendedchars=true,              % lets you use non-ASCII characters; for 8-bits encodings only, does not work with UTF-8
  frame=none,	                   % adds a frame around the code
  keepspaces=true,                 % keeps spaces in text, useful for keeping indentation of code (possibly needs columns=flexible)
  keywordstyle=\ttfamily\bfseries,       % keyword style
  otherkeywords={*,...},           % if you want to add more keywords to the set
  numbers=left,                    % where to put the line-numbers; possible values are (none, left, right)
  numbersep=5pt,                   % how far the line-numbers are from the code
  numberstyle=\footnotesize\color{mygray}, % the style that is used for the line-numbers
  rulecolor=\color{black},         % if not set, the frame-color may be changed on line-breaks within not-black text (e.g. comments (green here))
  showspaces=false,                % show spaces everywhere adding particular underscores; it overrides 'showstringspaces'
  showstringspaces=false,          % underline spaces within strings only
  showtabs=false,                  % show tabs within strings adding particular underscores
  stepnumber=1,                    % the step between two line-numbers. If it's 1, each line will be numbered
  stringstyle=\ttfamily,     % string literal style
  tabsize=4,	                   % sets default tabsize to 2 spaces
  title=\lstname                   % show the filename of files included with \lstinputlisting; also try caption instead of title
  % stringstyle=\color{mymauve}\ttfamily,     % string literal style
  % keywordstyle=\ttfamily\color{blue},       % keyword style
  % commentstyle=\ttfamily\color{mygreen},    % comment style
} % Файл со стилями

\begin{document}
\begin{titlepage}
    \begin{center}
        ФЕДЕРАЛЬНОЕ ГОСУДАРСТВЕННОЕ АВТОНОМНОЕ
        
        ОБРАЗОВАТЕЛЬНОЕ УЧРЕЖДЕНИЕ
        
        ВЫСШЕГО ОБРАЗОВАНИЯ
        
       <<НАЦИОНАЛЬНЫЙ ИССЛЕДОВАТЕЛЬСКИЙ УНИВЕРСИТЕТ
       
       ''ВЫСШАЯ~ШКОЛА~ЭКОНОМИКИ''>>
       \vspace{1cm}
 
        \textbf{Московский институт электроники и математики}
        
        Юткин Дмитрий Игоревич, группа БИВ-141
        \vspace{1cm}
        
        \textbf{\MakeUppercase{Применение глубоких свёрточных нейронных сетей для решения задачи распознавания образов}}
        \vspace{1cm}

        Междисциплинарная курсовая работа
        
        по направлению 09.03.01.62 Информатика и вычислительная техника 

        студента образовательной программы бакалавриата
        
        <<Информатика и вычислительная техника>>
        
    \end{center}
    \begin{flushright}
        Студент~\rule{4cm}{.1pt}~Д.И.\,Юткин
        \vspace{1cm}
        
        Научный руководитель

        Старший преподаватель

        Д.В.\,Пантюхин
        
        \rule{4cm}{.1pt}
    \end{flushright}
    \vfill\center{Москва 2016 г.}
\end{titlepage} % Титульник
\section*{Аннотация}
Работа посвящена решению задачи распознавания образов с использованием глубоких свёрточных 
нейронных сетей. В результате исследования было обучено несколько нейронных сетей различной глубины 
(от 9 до 18 слоёв), часть из которых обучалась на предварительно обработанных данных. Максимальная 
точность одиночной модели составила 93,4\%, объединение нейронных сетей в ансамбль позволило
увеличилась точность распознавания до 94,21\%, что сравнимо с точностью человека. Все эксперименты 
проводились на наборе изображений CIFAR-10, с использованием графических процессоров и фреймворка
для глубокого обучения Caffe.

\section*{Abstract}
The research focused on application of deep convolutional neural networks for
solving object recognition task in natural images. As a result, several models of different depth
(from 9 to 18 layers) have been trained on preprocessed and augmented data. The best model has shown accuracy 
of 93,4\% on a test images. Whereas, ensemble of combined models has achived 94,21\% which is 
close-to-human performance. All experiments have been carried out on the CIFAR-10 dataset with modern GPUs
and Caffe framework for a deep learning. % Аннотация
\tableofcontents
 % Аннотация  
\section{Введение}
В последние годы компьютерное зрение является одной из самых активно развивающихся областей 
искусственного интеллекта, основная задача которой "--- научить компьютеры воспроизводить 
зрительные способности человека или животных, например, распознавать и классифицировать образы.

До недавнего времени распознавание объектов на изображениях осуществлялось с помощью 
алгоритмов, для которых вручную приходилось проектировать признаки (feature engineering). 
Основной недостаток такого подхода "--- невозможность находить в изображении 
средние и многоуровневые абстракции, такие как части объекта или пересечения различных краёв. 
Кроме того, признаки созданные вручную для одного типа изображений зачастую не были применимы к 
другим. Однако, недавние разработки в области машинного обучения, известные как <<глубокое 
обучение>> (deep learning), показали, как вычислительные модели, состоящие из множества слоёв, 
могут автоматически изучать иерархии признаков прямо из набора данных.

На сегодняшний день глубокое обучение развилось и укрепилось в отдельную ветвь машинного обучения, 
алгоритмы которой способствовали значительному улучшению результатов в различных задачах, таких как 
обработка естественного языка, распознавание визуальных образов и др. Благодаря технологиям 
глубокого обучения сегодня мы можем искать похожие фотографии в Google Photos, или восхищаться 
беспилотными автомобилями, которые управляются искусственным интеллектом.

Свёрточные нейронные сети (СНН) "--- одна из главных причин прорыва в компьютерном зрении. 
Изначально представленные в 1980 году Кунихикой Фукусимой как NeoCognitron \cite{Neocognitron}, а 
затем улучшенные в 1998 году Яном Лекуном до LeNet-5 \cite{lecun-98}, СНН получили славу благодаря 
впечатляющему успеху в  распознавании рукописных цифр. Для следующего прорыва в компьютерном зрении 
потребовалось чуть больше двадцати лет. С увеличением производительности графических процессоров 
(GPU), стало возможным обучение глубоких нейронных сетей. В 2012 году глубокая СНН победила в 
мировом соревновании ImageNet Large-scale Visual Recognition Challenge (ILSVRC), значительно 
превзойдя предыдущие результаты, достигнутые алгоритмами полагающимися на ручную генерацию 
признаков \cite{NIPS2012_4824}.

Темой данного исследования является разработка системы распознавания объектов (object recognition), 
основанной на глубоких свёрточных нейронных сетях. Задача решалась с помощью  фреймворка для 
глубокого обучения Caffe \cite{jia2014caffe} на наборе данных CIFAR-10 \cite{learningmultiple}. 
Основные цели исследования заключались в следующем:
\begin{enumerate}
    \item Изучить и применить на практике свёрточные нейронные сети в решении задачи распознавания 
    объектов;
    \item Изучить современные технологии машинного обучения, анализа данных и глубокого  
    обучения.
\end{enumerate}
Для достижения поставленных целей были сформулированы следующие задачи:
\begin{enumerate}
    \item Провести обзор статей и литературы, посвящённых свёрточным нейронным сетям и решению 
    задачи распознавания образов;
    \item Выбрать программное обеспечение для обучения нейронных сетей;
    \item Спроектировать и обучить нейронные сети различных архитектур;
    \item Оценить и сравнить точности полученных нейронных сетей;
    \item Объединить отдельно обученные модели в ансамбль, с целью увеличения точности 
    распознавания.
\end{enumerate}
Все поставленные задачи выполнены, цели исследования достигнуты. Результаты каждой из задач будут 
описаны далее, в основной части работы.



 % Введение

\bibliography{biblio}

\end{document}