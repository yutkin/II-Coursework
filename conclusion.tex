\section{Заключение}
В результате выполнения курсовой работы был спроектирован и обучен ансамбль моделей,
состоящий из семи различных глубоких свёрточных нейронных сетей и решающий задачу распознавания
образов на датасете CIFAR-10 с точностью, превосходящей точность распознавания человека.

В процессе исследования были получены теоретические знания в области машинного обучения,
анализа данных и глубокого обучения. В частности, были подробно изучены свёрточные нейронные 
сети, способы их проектирования, настройки и оптимизации, а также фреймворк Caffe, позволяющий
реализовывать модели и современные алгоритмы компьютерного зрения. Полученные знания послужат 
фундаментом для дальнейших исследований в этой области. Таким образом, по результатам работы можно 
сказать, что все поставленные задачи выполнены, цели достигнуты.


